%%%%%%%%%%%%%%%%%%%%%%%%%%%%%%%%%%%%%%%%%
% Medium Length Professional CV
% LaTeX Template
% Version 2.0 (8/5/13)
%
% This template has been downloaded from:
% http://www.LaTeXTemplates.com
%
% Original author:
% Rishi Shah 
%
% Important note:
% This template requires the resume.cls file to be in the same directory as the
% .tex file. The resume.cls file provides the resume style used for structuring the
% document.
%
%%%%%%%%%%%%%%%%%%%%%%%%%%%%%%%%%%%%%%%%%

%----------------------------------------------------------------------------------------
%	PACKAGES AND OTHER DOCUMENT CONFIGURATIONS
%----------------------------------------------------------------------------------------

\documentclass{resume} % Use the custom resume.cls style

\usepackage[left=0.75in,top=0.6in,right=0.75in,bottom=0.6in]{geometry} % Document margins
\usepackage{hyperref}
\usepackage{fontawesome5}
\usepackage{amsmath}
\usepackage{enumitem}

\newcommand{\tab}[1]{\hspace{.2667\textwidth}\rlap{#1}}
\newcommand{\itab}[1]{\hspace{0em}\rlap{#1}}
\name{Arnav Sood} % Your name
\address{17 Sparrow Drive, Princeton Jct., NJ 08550} % Your address
\begin{document}

\faIcon{git-square} \href{https://github.com/arnavs}{\tt arnavs} \vline \hspace{0.5 em} {\faIcon{globe}} \href{https://arnavsood.com}{\tt arnavsood.com} \vline \hspace{0.5 em} {\faIcon{envelope}} \href{mailto:arnav@arnavsood.com}{\tt arnav@arnavsood.com} \vline \hspace{0.5 em} \faIcon{phone} (+1) 609.285.7001 \vline \hspace{0.5 em} \faIcon{passport} USA 

\begin{rSection}{Employment History}
Predoctoral Researcher, {\bf University of British Columbia} \hfill {Jun. 2018 --- Jun. 2020} 
\\ Advisor: Prof. Jesse Perla \smallskip 

Lead Developer, {\bf QuantEcon} \hfill {Jan. 2019 --- Present}
\\ References: Prof. John Stachurski, Dr. Matt McKay, Dr. Chase Coleman

Research Assistant, {\bf Prof. Laura Veldkamp} \hfill {May. 2016 --- Jan. 2018}

\end{rSection}

\begin{rSection}{Education}
{\bf University of British Columbia} \hfill {Jun. 2018 --- Jun. 2020} 
\\ Selected Courses (MS and BA.)  % \hfill { A/A+ Grades }

{\bf New York University} \hfill {Sep. 2014 --- May 2018} 
\\ Bachelor of Arts (Math.) % \hfill { GPA: 3.2 } 
\\ Minors in Economics, Philosophy 
\end{rSection}

\begin{rSection}{Publications}
    {\bf Exploiting Symmetry in High-Dimensional Dynamic Programming} \hfill {In-Progress} 
    \\ (with Jesse Perla, Mahdi Kahou, Jes\'{u}s Fern\'{a}ndez-Villaverde)    

    We provide a new method for solving high-dimensional dynamic programming problems, and recursive competitive equilibria with a very large (but finite) number of heterogenous agents.  The ``curse of dimensionality'' is avoided due to three complementary techniques: (1) exploiting symmetry in the approximate law of motion and the value function when designing deep learning approximations; (2) constructing a concentration of measure to calculate high-dimensional expectations using only a \textit{single} Monte-Carlo draw for all idiosyncratic shocks; and (3) sampling methods to ensure the model fits along manifolds of interest. 
\end{rSection}

\begin{rSection}{Research Assistantships and Software}
    {\bf \href{https://github.com/jlperla/PerlaTonettiWaugh.jl}{Equilibrium Technology Diffusion, Trade, and Growth}} % \hfill {Research Code}
    \begin{itemize}
        \item Co-wrote Julia code which solves a forward-looking differential system in steady-state, and computes transition dynamics in response to shocks.
    \end{itemize}

    {\bf \href{https://notes.quantecon.org/submission/5c832d2be7b4c5000f4c8e48}{Optimal Stopping and Linear Complementarity}} % \hfill {Co-Author}
    \begin{itemize}
        \item A short note with my advisor about how Optimal Stopping Problems (OSPs) can be reformulated as linear complementarity problems (LCPs), as opposed to the traditional Bellman-style approach.
    \end{itemize}

    {\bf \href{https://julia.quantecon.org}{QuantEcon Julia Lectures}} % \hfill {Co-Author}
    \begin{itemize}
        \item Wrote new lectures, overhauled code, deployed to cloud backends, and supervised RAs.
    \end{itemize}

    {\bf \href{https://github.com/quantecon/Expectations.jl}{QuantEcon/Expectations.jl}} % \hfill {Research Code}
    \begin{itemize}
        \item Uses Gaussian quadrature to take expectations for increased clarity, speed, and accuracy. Accepted as a poster at JuliaCon 2020.
    \end{itemize}

    {\bf \href{https://github.com/quantecon/InstantiateFromURL.jl}{QuantEcon/InstantiateFromURL.jl}} % \hfill {Research Code}
    \begin{itemize}
        \item Allows Julia Jupyter notebooks to run anywhere with proper package versions. Accepted as a talk at JuliaCon 2020 and used in QuantEcon lectures.
    \end{itemize}

    {\bf \href{https://vse.syzygy.ca}{VSE Syzygy JupyterHub}} % \hfill {Co-Author}
    \begin{itemize}
        \item Worked with Dr. Ian Allison of PIMS to maintain a JupyterHub server for faculty and student use. Deployed from Docker for reproducible setup.
    \end{itemize}
    
    

\end{rSection}

\end{document}
